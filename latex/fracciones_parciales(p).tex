\section*{Problemas}


\subsection*{Sistemas lineales}


\begin{problema}
	\begin{align*}
		2x-y&=4\\
		x+y&=5
	\end{align*}

\end{problema}




\begin{problema}
	\begin{align*}
		5x+2y&=3\\
		2x+3y&=-1
	\end{align*}

\end{problema}




\begin{problema}
	\begin{align*}
		2x+3y=3\\
		6y-6x=1
	\end{align*}

\end{problema}




\begin{problema}
	\begin{align*}
		5y&=3-2x\\
		3x&=2y+1
	\end{align*}

\end{problema}




\begin{problema}
	\begin{align*}
		\dfrac{x-2}{3}+\dfrac{y+1}{6}=2\\
		\dfrac{x+3}{4}-\dfrac{2y-1}{2}=1
	\end{align*}

\end{problema}

\subsection*{Regla de Cramer}


El m\'etodo de soluci\'on de sistemas de ecuaciones linales, por medio de determinantes, se conoce como Regla de Cramer.



\begin{problema} Resuelva el siguiente sistema por la Regla de Cramer
	\label{spi:28.4a}
	$$
	\begin{cases}
		4x+2y=5\\
		3x-4y=1
	\end{cases}
	$$
\end{problema}




\begin{problema} Resuelva el siguiente sistema por la Regla de Cramer
	\label{spi:28.4b}
	$$
	\begin{cases}
		3u+2v=18\\
		-5u-v=12
	\end{cases}
	$$
\end{problema}


\subsection*{Fracciones parciales}

%%%%%%%%%%%%%%%%%%%%%
{}
\begin{problema}
	Encuentre la expresión en fracciones parciales de
	\begin{align*}
		\dfrac{-29 \, x + 143}{2 \, x^{2} - 22 \, x + 56}
	\end{align*}
\end{problema}

\begin{align*}
	\dfrac{-29 \, x + 143}{2 \, x^{2} - 22 \, x + 56}= -\frac{9}{2 \, {\left(x - 4\right)}} - \frac{10}{x - 7}
\end{align*}


%%%%%%%%%%%%%%%%%%%%%
{}
\begin{problema}
	Encuentre la expresión en fracciones parciales de
	\begin{align*}
		\dfrac{-10 \, x - 30}{7 \, x^{2} + 4 \, x - 3}
	\end{align*}
\end{problema}

\begin{align*}
	\dfrac{-10 \, x - 30}{7 \, x^{2} + 4 \, x - 3}= -\frac{24}{7 \, x - 3} + \frac{2}{x + 1}
\end{align*}


%%%%%%%%%%%%%%%%%%%%%
{}
\begin{problema}
	Encuentre la expresión en fracciones parciales de
	\begin{align*}
		\dfrac{171 \, x + 175}{15 \, x^{2} + 38 \, x + 7}
	\end{align*}
\end{problema}

\begin{align*}
	\dfrac{171 \, x + 175}{15 \, x^{2} + 38 \, x + 7}= \frac{22}{5 \, x + 1} + \frac{21}{3 \, x + 7}
\end{align*}


%\subsection{Factores lineales con repetición}

%%%%%%%%%%%%%%%%%%%%%
{}
\begin{problema}
	Encuentre la expresión en fracciones parciales de
	\begin{align*}
		\dfrac{-10 \, x + 109}{x^{2} - 20 \, x + 100}
	\end{align*}
\end{problema}

\begin{align*}
	\dfrac{-10 \, x + 109}{x^{2} - 20 \, x + 100}= -\frac{10}{x - 10} + \frac{9}{{\left(x - 10\right)}^{2}}
\end{align*}


%%%%%%%%%%%%%%%%%%%%%
{}
\begin{problema}
	Encuentre la expresión en fracciones parciales de
	\begin{align*}
		\dfrac{80 \, x + 31}{256 \, x^{2} + 96 \, x + 9}
	\end{align*}
\end{problema}

\begin{align*}
	\dfrac{80 \, x + 31}{256 \, x^{2} + 96 \, x + 9}= \frac{5}{16 \, x + 3} + \frac{16}{{\left(16 \, x + 3\right)}^{2}}
\end{align*}


%%%%%%%%%%%%%%%%%%%%%
{}
\begin{problema}
	Encuentre la expresión en fracciones parciales de
	\begin{align*}
		\dfrac{30 \, x - 14}{25 \, x^{2} - 20 \, x + 4}
	\end{align*}
\end{problema}

\begin{align*}
	\dfrac{30 \, x - 14}{25 \, x^{2} - 20 \, x + 4}= \frac{6}{5 \, x - 2} - \frac{2}{{\left(5 \, x - 2\right)}^{2}}
\end{align*}


