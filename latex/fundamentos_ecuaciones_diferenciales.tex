\section{Fundamentos de Ecuaciones Diferenciales}

\subsection{Conceptos básicos}

{Definición}
	Una \emph{ecuación diferencial} es una ecuación que involucra derivadas o diferenciales de una o varias variables.

%%%%%%%
{}
	Si la ecuación sólo involucra derivadas respecto a una única variable independiente, diremos que es \emph{ordinaria}.  En otro caso, que es \emph{parcial}.

%%%%%%%
{Orden}
	Si la ecuación involucra derivadas de orden $n$, pero no de orden más alto, diremos que la propia ecuación es de \emph{orden $n$.}

%%%%%%%
{}
	\begin{problema}
		\label{exmp 02_01}
		\begin{align*}
		\left(y''\right)^{2}+3x=2\left(y'\right)^3
		\end{align*}
	\end{problema}

%%%%%%%

	\begin{problema}
		\label{exmp 02:02}
		\begin{align*}
		\dfrac{dy}{dx}+\dfrac{y}{x} = y^2
		\end{align*}
	\end{problema}

%%%%%%%
{}
	\begin{problema}
		\label{exmp 02:03}
		\begin{align*}
			\dfrac{d^{2}Q}{dt^{2}}-3\dfrac{dQ}{dt}+2Q = 4\sin(2t)
		\end{align*}
	\end{problema}

%%%%%%%
{}
	\begin{problema}
		\label{exmp 02:04}
		\begin{align*}
			\dfrac{dy}{dx}=\dfrac{x+y}{x-y}
		\end{align*}
		 
		De manera equivalente
		\begin{align*}
			(x+y)dx+(y-x)dy = 0 
		\end{align*}
	\end{problema}

%%%%%%%
{}
	\begin{problema}
		\begin{align*}
			\dfrac{\partial^{2}V}{\partial x^{2}} +
			\dfrac{\partial^{2}V}{\partial y^{2}} = 0
		\end{align*}
	\end{problema}

%%%%%%%
\subsection{Constantes arbitrarias}
{}
	Una constante arbitraria es un valor que es independiente de las variables involucradas en la ecuación.
	
	Generalmente las denotaremos con las primeras letras del alfabeto:
	\begin{align*}
		A,B,C,c_{1},c_{2},...
	\end{align*}

%%%%%%%%%%%
{}
	\begin{problema}
		En la ecuación
		\begin{align*}
		y = x^{2}+c_{1}x+c_{2}
		\end{align*}
		los símbolos $ c_{1}, c_{2} $ representan constantes arbitrarias. 
		
	\end{problema}

%%%%%%%
{}
	\begin{problema}
		La relación $ y = Ae^{-4x+B} $ se puede reescribir como $ y = Ce^{-4x} $.  Por lo que sólo involucra una constante arbitraria.
	\end{problema}

%%%%%%%

	Siempre reduciremos las ecuaciones al número mínimo de constantes arbitrarias, a las que llamaremos \emph{esenciales}.

%%%%%%%%%%
\subsection{Soluciones}
{}
	Una \emph{solución de una ecuación diferencial} es una relación entre las variables que está libre de derivadas, y que satisface la ecuación diferencial en al menos un intervalo.

%%%%%%%
{}
	Una \emph{solución general} de una ecuación diferencial de orden $ n $ es aquella que involucra $ n $ constantes arbitrarias esenciales.

%%%%%%%
{}
	\begin{problema}
		\label{exmp 02:06}
		$ y=x^{2}+c_{1}x+c_{2} $ es una solución general de $ y''=2 $.
	\end{problema}

%%%%%%%
{}
	Una \emph{solución particular} es aquella que se obtiene de una general, sustituyendo valores específicos en las constantes arbitrarias.

%%%%%%%
{}
	\begin{problema}
		\label{exmp 02:08}
		$ y = x^2-3x+2 $ es una solución particular de $ y''=2 $. 
	\end{problema}

%%%%%%%

	Una \emph{solución singular} es una aquella que no se puede obtener de la solución general sólo especificando valores para las constantes arbitrarias. 

%%%%%%%%
{}
	\begin{problema}
		La solución general de $ y = xy'-y'^{2} $ es $ y = cx-c^{2} $. 
		
		 Sin embargo, $ y=\dfrac{x^{2}}{4} $ es una solución que no se puede obtener sustituyendo $ c $.  Por tanto, es una solución particular.
	\end{problema}

%%%%%%%
\subsection{Ecuación diferencial de una familia de curvas}
%%%%%%%
{}
Una solución general de orden $ n $ tiene $ n $ parámetros (constantes arbitrarias esenciales) y por tanto, geométricamente representa una \emph{familia de curvas $n-$paramétrica. }

%%%%%%%
{}
	De manera reciproca, una relación con $ n $ constantes arbitrarias (también llamada \emph{primitiva}) tiene asociada una ecuación diferencial de orden $n$ (de la cual es solución general), llamada la \emph{ecuación diferencial de la familia}.

%%%%%%%
\subsection{Ejemplos}
{Clasificación de ecuaciones diferenciales}
\begin{problema}
		Clasifica cada una e las siguientes ecuaciones diferenciales enunciando su orden; sus variables dependientes e independientes; y si es ordinaria o parcial:	
\end{problema}

%%%%%%%
{}
		\begin{align*}
		x^2y''+xy'+\left(x^2-n^2\right)y = 0
		\end{align*} 
\begin{proof}[Solución]
	\begin{enumerate}[(i)]
		\item Orden 2;
		\item variable dependiente: $ y $;
		\item variable independiente: $ x $; 
		\item ecuación ordinaria.
	\end{enumerate}
\end{proof}

%%%%%%%
{}
	\begin{align*}
		\dfrac{dx}{dy}= x^{2}+y^{2}
	\end{align*} 
	\begin{proof}[Solución]
		\begin{enumerate}[(i)]
			\item Orden 1; 
			\item variable dependiente: $ x $; 
			\item variable independiente: $ y $; 
			\item ordinaria.
		\end{enumerate}
	\end{proof}

%%%%%%%
{}
	\begin{align*}
	\dfrac{dy}{dx}= \dfrac{1}{x^{2}+y^{2}}
	\end{align*} 
	\begin{proof}[Solución]
		\begin{enumerate}[(i)]
			\item Orden 1; 
			\item variable dependiente: $ y $; 
			\item variable independiente: $ x $; 
			\item ordinaria.
		\end{enumerate}
	\end{proof}

%%%%%%%
{}
	\begin{align*}
	\left(\dfrac{d^{2}u}{dt^{2}}\right)^{3}+u^{4}=1
	\end{align*} 
	\begin{proof}[Solución]
		\begin{enumerate}[(i)]
			\item Orden 2; 
			\item variable dependiente: $ u $; 
			\item variable independiente: $ t $; 
			\item ordinaria.
		\end{enumerate}
	\end{proof}

%%%%%%%

{}
	\begin{align*}
	\dfrac{\partial^{2}Y}{\partial t^{2}} = 2\dfrac{\partial^{2}Y}{\partial x^{2}}
	\end{align*} 
	\begin{proof}[Solución]
		\begin{enumerate}[(i)]
			\item Orden 2; 
			\item variable dependiente: $ Y $; 
			\item variables independientes: $ x,t $; 
			\item parcial.
		\end{enumerate}
	\end{proof}

%%%%%%%

{}
	\begin{align*}
	\left(x^{2}+2y^{2}\right)dx +\left(3x^{3}-4y^{2}\right)dy=0
	\end{align*} 
	\begin{proof}[Solución]
		\begin{enumerate}[(i)]
			\item Orden 1; 
			\item variable dependiente: $ y  $; 
			\item variable independiente: $ x $; 
			\item ordinaria.
		\end{enumerate}
	\end{proof}

%%%%%%%
{Solución de ecuaciones diferenciales}
	\begin{problema}
		Verifica para cada ecuación, si la relación indicada es solución; y en ese caso, determina si es general.
	\end{problema}

%%%%%%%
{}
	\begin{align*}
		\begin{cases}
		y'-x+y=0\\
		y = Ce^{-x}+x-1
		\end{cases}
	\end{align*}

%%%%%%%
{}
	\begin{proof}[Solución]
		\begin{enumerate}[(i)]
			\item $y'=-Ce^{-x}+1$  
			\item $y'-x+y= (-Ce^{-x}+1)-x+(Ce^{-x}+x-1)=0$ 
			\item $C$ es su único parámetro. 
			\item Por tanto $y$ es solución general.
		\end{enumerate}
	\end{proof}

%%%%%%%
{}
	\begin{align*}
		\begin{cases}
		\dfrac{dy}{dx}=\dfrac{2xy}{3y^{2}-x^{2}}\\
		x^{2}y-y^{3}=c
		\end{cases}
	\end{align*}

%%%%%%%
{}
	\begin{proof}[Solución]
		\begin{enumerate}[(i)]
			\item Derivando de forma implícita obtenemos
			\begin{align*}
				x^{2}y'+2xy-3y^{2}y'=0
			\end{align*}  
			\item Despejando $y'$ obtenemos
			\begin{align*}
				y'=\dfrac{2xy}{3y^{2}-x^{2}}
			\end{align*}  
			\item Como $C$ es el único parámetro, $y$ es una solución general. 
		\end{enumerate}
	\end{proof}

%%%%%%%
{}
	\begin{align*}
	\label{problem 2.2:c}
		\begin{cases}
		\dfrac{d^{2}I}{dt^{2}}+2\dfrac{dI}{dt}-3I =
		2\cos(t)-4\sin(t)\\
		I = c_{1}e^{t}+c_{2}e^{-3t}+\sin(t)
		\end{cases}
	\end{align*}

%%%%%%%
{}
	\begin{proof}[Solución]
		\begin{enumerate}[(i)]
			\item $ \dfrac{dI}{dt}
			=c_{1}e^{t}-3c_{2}e^{-3t}+\cos(t) $
			
			\item $ \dfrac{d^{2}I}{dt^{2}} = 
			c_{1}e^{t}+9c_{2}e^{-3t}-\sin(t) $ 
			 
			\item 
			\begin{align*}
			\left(c_{1}e^{t}+9c_{2}e^{-3t}-\sin(t)\right)\\
			+2\left(c_{1}e^{t}-3c_{2}e^{-3t}+\cos(t)\right)\\
			-3\left(c_{1}e^{t}+c_{2}e^{-3t}+\sin(t)\right)\\
			=   2\cos(t)-4\sin(t)
			\end{align*}
			 
			\item Como $ c_{1},c_{2} $ son parámetros, entonces $I$ es una solución general.
		\end{enumerate}
	\end{proof}

%%%%%%%
{}
	\begin{align*} 
	\begin{cases}	
	x^{3}\left(\dfrac{d^{2}v}{dx^{2}}\right)^{2} = 2v\dfrac{dv}{dx} \\
	v = cx^{2}
	\end{cases}
	\end{align*}

%%%%%%%
{}
	\begin{proof}[Solución]
		\begin{enumerate}[(i)]
			\item $ \dfrac{dv}{dx}=2cx $ 
			\item $ \dfrac{d^{2}v}{dx^{2}}=2c $  
			\item Sustituimos en el lado izquierdo
			\begin{align*}
				x^{3}\left(2c\right)^{2}=4c^{2}x^{3}
			\end{align*} 
			\item Sustituimos en el lado derecho
			\begin{align*}
				2\left(cx^{2}\right)\left(2cx\right)= 4c^{2}x^{3}
			\end{align*}  
			\item Entonces $v$ es solución, pero como la ecuación es de grado 2 y $c$ es el único parámetro, no es general. 
		\end{enumerate}
	\end{proof}

%%%%%%%
{}
  % PROBLEMA RESUELTO 2.3
  \begin{problema}
   Determine la solución particular de la ecuación diferencia del problema \ref{problem 2.2:c}, tal que satisface las condiciones 
     \begin{align*}
   I(0)&=2\\
   I'(0)&=-5
   \end{align*}
  \end{problema}


%%%%%%%%%%%%%%%%%%%%%
{}
      \begin{proof}[Solución]
    \begin{enumerate}[(i)]
      %NUEVO ITEM
      \item $I(t)=c_{1}e^{t}+c_{2}e^{-3t}+\sin(t)$ 
      \item $I(0)= c_{1}+c_{2}=  2$ 
      \item $I'(t)= c_{1}e^{t}-3c_{2}e^{-3t}+\cos(t)$ 
      \item $I'(0)= c_{1}-3c_{2}+1= -5$ 
      \item $c_{1}-3c_{2}=-6$ 
      \item $c_{1}=0, c_{2}=1$ 
      \item $I= 2e^{-3t}+\sin(t)$
\end{enumerate}
    \end{proof}

%%%%%%%%%%%%%%%%%%%%%
{}
  % problema resuelto 2.4
  \begin{problema}
   Mostrar que la solución de problema de valor inicial
       \begin{align*}
    \begin{cases}
Q''(t)+4Q'(t)+20Q(t)=16e^{-2t} & t\geq 0 \\
Q(0) = 2, Q'(0)=0 &
\end{cases}
    \end{align*}
    is 
         \begin{align*}
     Q(t) = e^{-2t}\left( 
     1+\sin(4t)+\cos(4t)
     \right)
     \end{align*}
  \end{problema}


%%%%%%%%%%%%%%%%%%%%%
{Solución}
     \begin{align*}
   Q'(t) &= e^{-2t}\left( 4\cos(4t)-4\sin(4t) \right) 
   -2e^{-2t}\left( 1+\sin(4t)+\cos(4t) \right)\\ 
   &=e^{-2t}\left( 2\cos(4t)-6\sin(4t)-2 \right)
   \end{align*}

%%%%%%%%%%%%%%%%%%%%%
{Solución}
  \begin{align*}
   Q''(t) &= e^{-2t}\left( -8\sin(4t)-24\cos(4t) \right)
   -2e^{-2t}\left( 2\cos(4t)-6\sin(4t)-2 \right)\\ 
   & = e^{-2t}\left( 4\sin(4t)-28\cos(4t)+4 \right)
  \end{align*}


%%%%%%%%%%%%%%%%%%%%%
{Solución}
     \begin{align*}
   Q''(t)+4Q'(t)+20Q(t) \\
   =  
   e^{-2t}\left( 4\sin(4t)-28\cos(4t)+4 \right) \\
   +4e^{-2t}\left( 2\cos(4t)-6\sin(4t)-2 \right) \\
   +20e^{-2t}\left( 
     1+\sin(4t)+\cos(4t)
     \right) \\
     =  16e^{-2t}
   \end{align*}


%%%%%%%%%%%%%%%%%%%%%
{}
  Determine gráficamente una relación entre la solución general 
     \begin{align*}
   y = cx-c^{2}
   \end{align*}
   y la solución singular $y=\dfrac{x^{2}}{4}$ de la ecuación diferencial 
       \begin{align*}
    y = xy'-y^{2}
    \end{align*}

%%%%%%%%%%%%%%%%%%%%%
{}
  \begin{figure}
 \centering
 \includegraphics[height=.7\textheight,keepaspectratio=true]{./edo/solved_problem_02-05.png}
 \caption{La parábola es la envolvente de la familia de lineas rectas. }
 % solved_problem_02-05.png: 587x387 px, 100dpi, 14.91x9.83 cm, bb=0 0 423 279
 \label{fig:solved_problem_02-05}
\end{figure}


%%%%%%%%%%%%%%%%%%%%%

La envolvente de una familia de curvas 
 \begin{align*}
 G(x,y,c)=0, 
 \end{align*}
 si es que existe, puede encontrarse resolviendo simultaneamente las ecuaciones
   \begin{align*}
  \begin{cases}
\partial_{c}G(x,y,c) = 0\\
G(x,y,c) = 0
\end{cases}
  \end{align*}


%%%%%%%%%%%%%%%%%%%%%
{Solución}
\begin{enumerate}[(i)]
  %NUEVO ITEM
  \item Calculamos la parcial $
   \partial_{c}G(x,y,c) =-x+2c$
 
 \item Plantemos las ecuaciones
   \begin{align*}
  -x+2c&=0\\
  y-cx+c^2&=0
  \end{align*}
  
  \item Resolvemos las ecuaciones y obtenemos la solución paramética
     \begin{align*}
   x=2c, y =c^{2}
   \end{align*}
   
   \item La solución se puede reescribir como
       \begin{align*}
    y = \dfrac{x^2}{4}
    \end{align*}
\end{enumerate}

%%%%%%%%%%%%%%%%%%%%%

%DIFFERENTIAL EQUATION OF A FAMILY OF CURVES, P. 46
{}
  \begin{problema}
   \begin{enumerate}
     %NUEVO ITEM
     \item Trace la gráfica de varios miembros de la familia uniparamétrica $$\sett{y=cx^2}{c\in \R}$$ 
     \item Obtenga la ecuación diferencial de esta familia
\end{enumerate}
  \end{problema}


%%%%%%%%%%%%%%%%%%%%%
{Solución: Inciso (a)}
  \begin{figure}
 \centering
 \includegraphics[height=.5\textheight,keepaspectratio=true]{./edo/solved_problem_02-06.png}
 % solved_problem_02-06.png: 587x387 px, 100dpi, 14.91x9.83 cm, bb=0 0 423 279
 \caption{Familia uniparamétrica $y=cx^2$}
 \label{fig:solved0206}
\end{figure}


%%%%%%%%%%%%%%%%%%%%%
{Solución: Inciso (b)}
  \begin{enumerate}[(i)]
    %NUEVO ITEM
    \item  $y=cx^3 \imply y'=3cx^{2}$ 
    \item $c=\dfrac{y}{x^3}$ 
    \item $y'=3\left( \dfrac{y}{x^3} \right)x^2$ 
    \item $y'=\dfrac{3y}{x}$
\end{enumerate}

%%%%%%%%%%%%%%%%%%%%%
{}
  \begin{problema}

  Encuentre una ecuación diferencial para la familia biparamétrica de cónicas 
         \begin{align*}
     ax^{2}+by^{2}=1
     \end{align*}

  \end{problema}


%%%%%%%%%%%%%%%%%%%%%
{}
      \begin{proof}[Solución]
      
      Supongamos que $b\neq 0$. 
    \begin{enumerate}[(i)] 
      %NUEVO ITEM
      \item $2ax+2byy'=0$ 
      \item $a = \dfrac{-byy'}{x}$ 
      \item $\left( -\dfrac{byy'}{x} \right)x^{2}+by^{2}=1$ 
      \item $-bxyy'+by^{2}=1$ 
      \item $-b\left( xyy''+xy'^{2}+yy' \right)
      +2byy'=0$ 
      \item $xyy''+xy'^{2}-yy'=0$
\end{enumerate}
    \end{proof}

%%%%%%%%%%%%%%%%%%%%%
{}
  \begin{enumerate}
    %NUEVO ITEM
    \item  Encuentra una solución general para la ecuación diferencial $\dfrac{dy}{dx}=3x^{2}$
    \item Traza la gráfica de las soluciones obtenidas en el inciso (a)
    \item Determina la ecuación de la curva particular en el inciso (b), que pasa por el punto $\left( 1,3 \right)$
\end{enumerate}

%%%%%%%%%%%%%%%%%%%%%
{Solución: Inciso (a)}
 \begin{enumerate}[(i)]
   %NUEVO ITEM
   \item $dy=3x^{2}dx$ 
   \item $\displaystyle 
   \int dy = \int 3x^{2}dx$
   
   \item $y=x^{3}+c$
\end{enumerate}

%%%%%%%%%%%%%%%%%%%%%
{Solución: Inciso (b)}
  \begin{figure}
 \centering
 \includegraphics[height=.6\textheight,keepaspectratio=true]{./edo/solved_problem_02-08.png}
 % solved_problem_02-08.png: 630x470 px, 100dpi, 16.00x11.94 cm, bb=0 0 454 338
 \caption{Familia uniparamétrica $y=x^3+c$}
 \label{fig:solved_problem_02-08}
\end{figure}


%%%%%%%%%%%%%%%%%%%%%
{Inciso (c)}
  \begin{enumerate}[(i)]
    %NUEVO ITEM
    \item Como la curva pasa por $(1,3)$, entonces
    $$x=1 \imply y =3$$
    \item $$
    3 = 1^{3}+c \imply c=2
    $$ 
    \item $y=x^{3}+2$
\end{enumerate}

%%%%%%%%%%%%%%%%%%%%%
{}
  \begin{problema}
   Resuelva el problema de condición inicial 
 \begin{align*}
 \begin{cases}
y''=3x-2\\
y(0)=2\\
y'(1)=-3
\end{cases}
 \end{align*}
  \end{problema}

%%%%%%%%%%%%%%%%%%%%%
{}
      \begin{proof}[Solución]
    \begin{enumerate}[(i)]
      %NUEVO ITEM
      \item $y'=\dfrac{3x^{2}}{2}-2x+c_{1}$ 
      \item $y'(1)=-3\imply      
      -3=\dfrac{3}{2}-2+c_{1} \imply       
      c_{1}=-\dfrac{5}{2}$      
      \item $y= \dfrac{x^{3}}{2}-x^{2}-\dfrac{5x}{2}+c_2$      
      \item 
      $y(0)=2\imply       
      c_{2}=2 \imply      
      y = \dfrac{x^{3}}{2}-x^{2}-\dfrac{5x}{2}+2$
\end{enumerate}
    \end{proof}

%%%%%%%%%%%%%%%%%%%%%
