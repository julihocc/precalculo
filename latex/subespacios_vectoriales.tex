\section{Subespacios vectoriales}



\begin{problema}
 $\R^{2}$ es un espacio vectorial con las operaciones $$(u_{1},u_{2})+(v_{1},v_{2})=(u_{1}+v_{1},u_{2}+v_{2})$$ y
$$\a(u_{1}, u_{2})=(\a u_{1}, \a u_{2}).$$

En la secci\'on anterior consideramos el subjunto $$L_{c}=\set{(u_{1},u_{2})|u_{2}=c u_{1}}\subset \R^{2}$$ para  $c$
una pendiente fija, y verificamos que en efecto, con las mismas operaciones es un espacio vectorial.

Decimos entonces que $L_{c}$ es un subespacio vectorial de $\R^{2}.$
\end{problema}

\begin{definicion}
 Si $(V, +, \cdot )$ es un espacio vectorial  y
$W\subset V$ es tambi\'en espacio vectorial, con las mismas operaciones $+,\cdot$ decirmos que $W$ es un subespacio
vectorial de $V,$ y podemos escribir $W < V.$
\end{definicion}

En principio, si $W < V$, tendr\'iamos que verificar todos loa axiomas de espacio vectorial para $(W, +, \cdot).$ Sin
embargo, si en el espacio $V,$ la suma es asociativa y conmutativa, tambi\'en lo será en $W.$ De igual manera, el
elemento neutro $1\in V$ de la multiplicaci\'on por escalares es el mismo en $W,$ y se sigue cumpliendo la
asociatividad de la multiplicaci\'on por escalares y las leyes de distribuci\'on.

Entonces, basta demostrar que se cumplen los restantes axiomas, a saber:
\begin{enumerate}
 \item Si $u,v \in W,$ entonces $u+v\in W.$
 \item Si $\a \in \R, v \in W,$ entonces $\a v\in W.$
 \item $0\in W.$
 \item Si $u\in W,$ entonces $-u\in W.$
\end{enumerate}

Sin embargo, los dos últimos incisos se siguen del segundo. En efecto, si escogemos $\a=0$ y cualquier $u\in
W,$ entonces
$$
0=0\cdot u \in W.
$$
De igual manera, para cualquier $u\in W,$ si escogemos $\a=-1,$ entonces $-u=(-1)u\in W.$

Por último, verificar los dos axiomas restantes es equivalente a verificar que para todo $\a \in \R, u,v \in W,$
$$
\a u + v \in W.
$$

\begin{proposicion}
 Si $W\subset V,$ entonces $$
W < V \iff \forall \a \in \R, u,v \in W, \a u+v\in W.
 $$
\end{proposicion}

\begin{corolario}
 Todo $W < V$ contiene a $0 \in V.$
\end{corolario}


\begin{definicion}
 Si $W< V,$ pero $W\neq\set{0}$ y $W\neq V,$ entonces decimos que $W$ es un subespacio (vectorial) propio.
\end{definicion}

\begin{definicion}
 Sean $u,v_{1},...,v_{k}$ vectores en un espacio vectorial $V.$ Decimos que $u$ es combinaci\'on lineal de
$v_{1},...,v_{k}$ si existen escalares $\a_{1},...,\a_{k}\in \R$ tales que:
$$
u=\a_{1}v_{1}+...\a_{k}v_{k}.
$$
\end{definicion}

\begin{definicion}
 Sea $V$ un espacio vectorial. El subespacio generado por un subconjunto $S=\set{v_{1},...,v_{k}}\subset V$ se define
como
 $$
\gen{S}=\set{\a_{1}v_{k},...,\a_{k}v_{k}\av \a_{1},...,\a_{k}\in \R},
 $$
 es decir, el conjunto de todas las combinaciones lineales de $v_{1},...,v_{k}.$
\end{definicion}

\begin{observacion}
 $\gen{S}<V.$
\end{observacion}


\begin{problema}
 $u=(2,0,2)$ es combinaci\'on lineal de $v_{1}=(1,0,1)$ y $v_{2}=(0,1,1)$ porque $u=2v_{1}-v_{2}.$

 De hecho, $$\gen{v_{1},v_{2}}=\set{(a,b,a+b)\av a,b \in \R}<\R^{3}$$ es el plano que contiene a estos dos vectores.

 $(-1,-1,1)\notin \gen{v_{1},v_{2}},$ porque no vive en este plano.
\end{problema}





\subsection*{Ejemplos}

%Los siguientes ejercicios se pueden encontrar en \cite[sec. 4.3]{G} y \cite[sec. 2.2]{HK}.

% \begin{problema} En los siguientes ejercicios, verificar que $W < V,$ con las operaciones $+,\cdot$ usuales en $V.$
% Justifique su respuesta.
%  \begin{enumerate}
%   \item $V=\R^{2}, W=\set{(x,y)|ax+by=0},$ con $a,b\in \R$ fijos
%   \item $V=\R^{3}, W=\set{(at,bt,ct)|t\in \R},$ con $a,b,c\in \R$ fijos
%   \item $V=\R^{3}, W=\set{(a,y,z)|ax+by+cz=0},$ con $a,b,c\in \R$ fijos
%   \item $V=P_{n}, W=P_{m},$ donde $P_{k}$ es el espacio de polinomios (con coeficientes reales) de grado menor
% o igual a $k$ y $n\geq m.$
% \item $V=M_{mn}, W=\set{(a_{ij})\in M_{mn}|a_{11}=0}$
% \item $V=C[0,1], W=P_{n}[0,1],$ donde $P_{k}$ es el espacio de polinomios (con coeficientes reales) de grado
% menor o igual a $k$ restringidos al intervalo $[0,1].$
% \item $V=C[0,1], W=\set{f\in V|\int_{0}^{1}f(x)dx=0}$
% 
%  \end{enumerate}
% \end{problema}

\begin{problema}
 ¿Cuál de los siguientes conjuntos de vectores $$u=(u_{1},u_{2},u_{3})$$ en $\R^{3}$ son subespacios de $\R^{3}$?
 \begin{enumerate}
  \item $\set{u\av u_{1}\geq 0}$
  \item $\set{u\av u_{1}+3u_{2}=u_{3}}$
  \item $\set{u\av u_{2}=u_{1}^{2}}$
  \item $\set{u\av u_{1}u_{2}=0}$
  \item $\set{u\av a_{2}\text{ es racional}}$
 \end{enumerate}

\end{problema}

\begin{problema}
 Sea $V$ el espacio vectorial (real) de todas las funciones $f:\R \to \R.$ ¿Cuales de los siguientes conjuntos son
subespacios de V?
\begin{enumerate}
 \item todas las funciones $f$ tales que $f(x^{2})=f^{2}(x)$
 \item todas las funciones $f$ tales que $f(0)=f(1)$
 \item todas las funciones $f$ tales que $f(3)=1+f(-5)$
 \item todas las funciones $f$ tales que $f(-1)=0$
\end{enumerate}

\end{problema}

% \begin{problema} ¿Es cierto que
% $$(3,-1,0,-1)\in \gen{(2,-1,3,2), (-1,1,1,-1), (1,1,9,-5)}$$?
% \end{problema}

\begin{problema}
 Sea $W$ el conjunto de todos los vectores $(x_{1},...,x_{5})$ que satisfacen
 \begin{align*}
  2x_{1}-x_{2}+\frac{4}{3}x_{3}-x_{4}&=0 \\
  x_{1}+\frac{2}{3}x_{3}-x_{5}&=0 \\
  9x_{1}-3x_{2}+6x_{3}-3x_{4}-3x_{5}&=0.
 \end{align*}
Mostrar que $W$ es subespacio vectorial de $\R^{5}.$
\end{problema}



\begin{problema}[\dag]
\begin{enumerate}
   \item  Verificar que si $U,W < V,$ entonces $U\cap W < V.$
   \item Demostrar que si $U=\set{u_{1},...,u_{m}}$ y $W=\set{w_{1},...,w_{m}},$ entonces
   $$
U\cap W = \gen{u_{1},...,u_{n},w_{1},...,w_{m}}.
   $$
\end{enumerate}


\end{problema}

% \begin{problema}
%  En los siguientes ejercicios, verificar \emph{si} $W < V,$ con las operaciones $+,\cdot$ usuales en $V.$
% Justifique su respuesta.
% \begin{enumerate}
% \item $V=M_{nn}, W=\set{A\in M_{nn}|A\text{ es invertible}}$
% \item $V=M_{nn}, W=\set{A\in M_{nn}|A\text{ no es invertible}}$
% \item $V=M_{nn}, W=\set{A\in M_{nn}|AB=BA}$ para alguna matriz $B\in M_{nn}$ fija.
% \item $V=M_{nn}, W=\set{A\in M_{nn}|A^{2}=A}$
% \item $V=M_{nn}, W=\set{A\in M_{nn}|A\text{ es diagonal}}$ \footnote{Consulte \cite[pag. 114]{G}}
% \item $V=M_{nn}, W=\set{A\in M_{nn}|A\text{ es triangular}}$ \footnote{Consulte \cite[pag. 88]{G}}
% \item $V=M_{nn}, W=\set{A\in M_{nn}|A\text{ es simetrica}}$ \footnote{Consulte \cite[pag. 123]{G}}
%  \item $V=C^{1}[0,1], W=\set{f\in C^{1}[0,1]|f'(0)=0}$
%  \item $V=C[a,b], W=\set{f\in C[a,b]| \int_{a}^{b}f(x)dx=0}$
%  \item $V=C[a,b], W=\set{f\in C[a,b]| \int_{a}^{b}f(x)dx=0}$
%  \end{enumerate}
% 
% 
% \end{problema}


\begin{problema}[\dag]
\begin{itemize}
 \item Sean $u,v \in \R^{2}.$ Mostrar que $$W=\set{\a u+ \b v| \a, \b \in \R} < \R^{2}.$$
 \item Mostrar que si $u,v$ no son paralelos, entonces para cualquier $w\in \R^{2},$ podemos encontrar $\a,\b \in
\R$ de manera que $w= \a u + \b v.$
\end{itemize}
\end{problema}

\begin{problema}[\dag]
 Sea $A$ una matriz $m\times n$ con entradas reales. Demostrar que el conjunto de todas los vectores columna $u$ de
longitud $n,$ tales que $Au=0$ es un subespacio vectorial de todos los vectores columna $\R^{n}.$
\end{problema}
